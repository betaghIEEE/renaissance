
\begin{abstract}
The most important questions in computer science are how quickly can a computation be performed, and how accurate are the results.  This thesis answers those questions in regard to a preconditioning for matrix multiplication.  In particular, this scheme is the Haar Wavelet Transform for Matrix Multiplication.  

Matrix multiplication itself is one of the more fundamental operations known to linear algebra, and computational scientist have been trying to shave the efficency of matrix multiply from its defining algorithm's $O(N^3)$ to something closer to $O(N^2)$.  Both sparse and non-sparse means have been tried in an effort to reduce computations.  

This Haar Wavelet Transform for Matrix Multiplication represents a preconditioning agent that when applied transforms a dense matrix into something more sparse.   This thesis shows how this transformation can be performed, and how accurate it, and how precise the result is even with the lower energy elements discarded.  
\end{abstract}