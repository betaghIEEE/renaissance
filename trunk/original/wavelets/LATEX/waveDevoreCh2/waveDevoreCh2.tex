
\documentclass{article}
%%%%%%%%%%%%%%%%%%%%%%%%%%%%%%%%%%%%%%%%%%%%%%%%%%%%%%%%%%%%%%%%%%%%%%%%%%%%%%%%%%%%%%%%%%%%%%%%%%%%%%%%%%%%%%%%%%%%%%%%%%%%
\usepackage{graphicx}
\usepackage{amsmath}

%TCIDATA{OutputFilter=LATEX.DLL}
%TCIDATA{Created=Tue Feb 25 21:06:46 2003}
%TCIDATA{LastRevised=Thu Feb 27 14:56:18 2003}
%TCIDATA{<META NAME="GraphicsSave" CONTENT="32">}
%TCIDATA{<META NAME="DocumentShell" CONTENT="General\Blank Document">}
%TCIDATA{CSTFile=LaTeX article (bright).cst}

\newtheorem{theorem}{Theorem}
\newtheorem{acknowledgement}[theorem]{Acknowledgement}
\newtheorem{algorithm}[theorem]{Algorithm}
\newtheorem{axiom}[theorem]{Axiom}
\newtheorem{case}[theorem]{Case}
\newtheorem{claim}[theorem]{Claim}
\newtheorem{conclusion}[theorem]{Conclusion}
\newtheorem{condition}[theorem]{Condition}
\newtheorem{conjecture}[theorem]{Conjecture}
\newtheorem{corollary}[theorem]{Corollary}
\newtheorem{criterion}[theorem]{Criterion}
\newtheorem{definition}[theorem]{Definition}
\newtheorem{example}[theorem]{Example}
\newtheorem{exercise}[theorem]{Exercise}
\newtheorem{lemma}[theorem]{Lemma}
\newtheorem{notation}[theorem]{Notation}
\newtheorem{problem}[theorem]{Problem}
\newtheorem{proposition}[theorem]{Proposition}
\newtheorem{remark}[theorem]{Remark}
\newtheorem{solution}[theorem]{Solution}
\newtheorem{summary}[theorem]{Summary}
\newenvironment{proof}[1][Proof]{\textbf{#1.} }{\ \rule{0.5em}{0.5em}}
%\input{tcilatex}

\begin{document}


Devore and Lucier make some simple arguments for the Haar Wavelet, and in
the treatment of wavelets in general. \ While they admit Haar Wavelets can
lack smoothness, they state the Haar Wavelet demonstrate the key features of
wavelet decomposition. \ \ \ Their argument:

\qquad 1. \ Let t be the independent variable.

\qquad 2. Let $H=\chi _{\lbrack 0,\frac{1}{2}]}-\chi _{\lbrack \frac{1}{2}%
,1]}=\left\{ 
\begin{tabular}{ll}
$1$ & $0\leq $t$<\frac{1}{2}$ \\ 
$-1$ & $\frac{1}{2}\leq $t$<1$ \\ 
$0$ & otherwise
\end{tabular}
\right. $

\qquad 3. Let the translation and dialation function be

\qquad \qquad \qquad $H_{j,k}(t)=2^{\frac{k}{2}}(2^{k}t-t)=2^{k/2}\left\{ 
\begin{tabular}{ll}
$1$ & $j2^{-k}\leq $t$<(j+\frac{1}{2})2^{-k}$ \\ 
$-1$ & $(j+\frac{1}{2})2^{-k}\leq $t$<(j+1)2^{-k}$ \\ 
$0$ & otherwise
\end{tabular}
\right. $

\qquad 4. \ What about $<H_{j,k},H_{j^{\prime },k^{\prime }}>$

\qquad \qquad a. \ Case $H_{j,k}$ and $H_{j^{\prime },k^{\prime }}$ are
disjoint $<H_{j,k},H_{j^{\prime },k^{\prime }}>=0$

\qquad \qquad b. \ Otherwise: \ $<H_{j,k},H_{j^{\prime },k^{\prime }}>=0$

\qquad 5. \ \{$H_{j,k}|j,k\in Z$\}, the Haar basis, is $L_{2}$ complete. \
Each $S=S^{0}\subset L_{2}(R)$ has a unique representation for dyadic shift
intervals $j2^{-k}$.

\qquad 6. \ Also, $S^{\infty }=L_{2}(R)$ and $S^{-\infty }=\{0\}$ .

\bigskip

There is an exercise in proving this argument. \ 

\bigskip

The main point of both forms of H ($H$ and $H(t-k)$) form an orthonormal
basis for W, which denotes wavelet space.

\bigskip

The Haar Decomposition and Smoothness space tends to be less understood. \
The whole point to the Haar Decomposition section is whether a wavelet can
be $L_{p}$ complete for an arbitrary $p\in \lbrack 1,\infty )$. \ The answer
is that for $p\in \lbrack 2,\infty )$, this is true. \ The $L_{1}$ case
shows the Haar Wavelet growing indefinately in energy. \ \ However, there is
a reference to special cases in ''Hardy space.'' \ \ At this time, the term
itself is being researched as an educational exercise.

\bigskip

One question, what is the relationship between smoothness of f and the size
of the Haar coefficents dependent on?

\qquad 1. \ For a fixed $k\in Z$ the Haar functions $(H_{j,k}),j\in Z$ are
stable.

\qquad 2. \ The Jackson inequality (How well functions from $Lip(\alpha
,L_{p}(R))$ can be approximated.

\qquad 3. \ The Bernstein Inequality ( Inequalities for Polynomials)

\bigskip

Another useful exercise is the proof of the Jackson and Bernstein
Inequalities.

\bigskip

The fast wavelet transform is so far the same, a convolution scheme. \ This
convolution scheme tosses out half of the values to produce convergence.

\bigskip

The mulivariate (multi-dimensional ) Haar Wavelet Functions have the two
methods:

\qquad 1. Sum of a wavelet Polynomial, which is the collection of functions $%
2^{kd/2}\psi _{v}(2^{k}\cdot -j),j\in Z^{d},k\in Z,v\in V\backslash \{0\}$

\qquad 2. The collection of all shifts of $\chi _{\lbrack 0,1]^{d}}$, which
is the tensor product of the univariate space of piecewise functions with
integer breakpoints.

\bigskip 

Belkin

\end{document}
