One overwhelming question drives computational science, how fast can the answer be computed?  Of course, the answer has to answer its own question.  In this thesis, the questions are for three computational areas of mathematics.  One is how fast can matrix multiplication be computed? Another is how fast can a matrix be inverted?  Third is how quickly can a partial differential equation be solved?  Of course there are already conventional algorithms to compute them, so where does this thesis come in?

All of these problems have one thing in common.  Each of these problems in the conventional methods are mostly solved via a matrix.  Obviously matrix multiplication and matrix inversion are classic matrix operations that obey the rules of linear algebra.  Partial differential equations have several varieties of methods which obtain solutions via a solution matrix.  

The wavelet transform is a linear orthonormal operation.   The three mathematical operations are solved by linear algebra type methods.   What is important about wavelets are the two most desired qualities in computation of matrices and systems of equations,  sparseness and condition number.  
Sparseness applied to matrix states a good majority of the elements in such a matrix are zero.   Condition number determines how quickly a matrix solution will converge.  The two are important due to the number one question in computational science as applied these mathematical problems, how fast can wavelet linear algebra be performed.  

In this thesis, an overview is provided to describe to the reader what wavelets are and their generic applications.  This overview covers a little image processing since it is the easiest means to show the qualities of a two-dimensional wavelet transform.  The three chapters that follow are devoted to matrix multiplication, matrix inversion and solutions to partial differential equations in respective order.  Those three chapters also discuss how wavelets can be used to solve those problems.  Furthermore, this these is dedicated to show how efficient wavelets are at solving these mathematical problems.  

%Wavelets in General

%Wavelets in matrix multiplication

%Wavelets in matrix inversion

%Wavelets in partial differential equations.