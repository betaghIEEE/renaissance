\documentclass{article}
%%%%%%%%%%%%%%%%%%%%%%%%%%%%%%%%%%%%%%%%%%%%%%%%%%%%%%%%%%%%%%%%%%%%%%%%%%%%%%%%%%%%%%%%%%%%%%%%%%%%%%%%%%%%%%%%%%%%%%%%%%%%
\usepackage{graphicx}
\usepackage{amsmath}
\usepackage{doublespace}

\begin{document}

The main things Dr. Beylkin highlighted in Wavelets and Fast Numerical
Algorithms were:
\begin{itemize}
\item All transform methods expand vectors and operators are expanded into a
basis and the computations take place in the new system of coordinates.
\item Typically, the choice of the differential operator, and the basis
functions are dictated by the availability of fast algorithms for expanding an
arbitrary function into the basis.
\item Representations in wavelet bases reduce a wide class of operators to a
sparce form.
\end{itemize}

The key word here is a "sparce" system.  Sparce is defined as having few
elements typically where there is concentration of the bulk of the elements and
the rest are insignificant.  

There are ingredientss of Calderon-Zygmund theory appear in the Fast Multipole
Method for computing potential interactions.  

Fast Wavelet Transforms provide a system generalization of the FMM and its
decendents to all Calderon-Zygmund and differential operators.  

\subsection{Non-standard form characteristics}
\begin{itemize}
\item uncoupling of interaction between the scales
\item explicit computation of basic operators such as derivatives, fractional
deriviatives, Hilbert and Riesz transforms.  
\item Solutions to two-point boundary value problem for ellipic differential
operators.  
\end {itemize}

\subsection{Multi-Resolution Analysis and Wavelet Reference Properties}

Definition: A multi-resolution analysis is a decomposition of the Hilbert space
$L^2 (R^d )$, , $d >= 1$ , into a chain of closed subspaces.

%$...\sub V_2 \sub V_1 \sub V_0 \sub V_{-1} \sub V_{-2} \sub ...$

Let $w_j$ be an orthogonal complement of $V_j$ in $V_{j-1}$ such that

$V_{j-1} = V_j +W_j$

and represent $L^2(R^d)= \sum _{j\in Z} W_j $  as a direct sum.

Consequences of Definition:
\begin{itemize}
\item The function may be described as linear combination of the basis function.
\item Orthogonality is unity, in the power domain, and does not add or remove
power from the original.  
\end {itemize} 

There are several mathematical transformation to illustrate these points:
\begin{itemize}
\item $\chi$ forms an orthonormal basis for V $(\chi (x-k)) \forall k \in Z$
\item $\psi$ forms an orthonormal basis for W $(\psi (x-k)) \forall k \in Z$

\end{itemize}

Condition for exact reconstruction for a pair of the quadrature mirror filters:

It may be a good idea to work out this proof and show results.  


\end{document}
