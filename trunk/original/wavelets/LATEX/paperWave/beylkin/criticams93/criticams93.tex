\documentclass[11pt]{article}
\usepackage{graphicx}
\usepackage{amssymb}
\usepackage{epstopdf}
\usepackage{doublespace}
\DeclareGraphicsRule{.tif}{png}{.png}{`convert #1 `basename #1 .tif`.png}

\textwidth = 6.5 in
\textheight = 9 in
\oddsidemargin = 0.0 in
\evensidemargin = 0.0 in
\topmargin = 0.0 in
\headheight = 0.0 in
\headsep = 0.0 in
\parskip = 0.2in
\parindent = 0.0in

\newtheorem{theorem}{Theorem}
\newtheorem{corollary}[theorem]{Corollary}
\newtheorem{definition}{Definition}

\title{Critic of AMS-93}
\author{Daniel D. Beatty}
\begin{document}
\maketitle

Watch out for 2 significant numerical methods 
\begin{enumerate}
\item Derivation: Differential Equations 
\item SVD
\end{enumerate}

Start with analysis notes and notes on Functional Analysis.

\section {Topological Vector Spaces 10 April 2003}
Functional Analysis by Rudin 
Supposition:  Suppose $\tau$ is a topology on a vector space X such that 
\begin{itemize}
\item every point of X is a closed set.
\item the vector space operations are continuous with respect to $\tau$.
\end{itemize}

To say that addition is continuous means, by definition that the mapping $(x,y) \rightarrow x+y$ of the cartesian product $X*X$ into X is continuous.  

Invariance: Let X be a topological vector space.  Associate to each $a\in X$ and to each scalar $\lambda \not= 0$ the translation operator $m_{\lambda}$ by the formula:

\begin{itemize}
\item $T_a (x)=x+a$
\item $M_{\lambda} (x) = \lambda x$
\end{itemize}

Both $T_a$ and $M_{\lambda}$ are homomorphisms on X onto X.

Types of Topological Vector Spaces

\section{Main Points of Wavelets and Fast Numerical Algorithms}

\begin{enumerate}
\item All transform methods expand vectors and operators are expanded into a basis and computations take place in the new system of coordinates.
\item Typically, the choice of the differential operator and, hence of the basis functions, is dictated by the availability of fast algorithms for expanding an arbitrary functions into the basis.
\item Representations in wavelet bases reduce a wide class of operators to a sparse form.
\end{enumerate}

One key term: Sparse system
A sparse system is a system with few or scattered elements.  

Ingredients of Calderon-Zygmund Theory appear in the Fast Multipole Method for computing potential interactions.  

The Fast Wavelet Transform provide a system generalization of the FMM and its descendants to all Calderon-Zygmund and differential operators.

Non-standard form characteristics 
\begin{itemize}
\item Uncoupling of interaction between the scales.
\item Explicitly computation of basic operators such as derivatives, Hilbert and Riesz transforms.  
\item Solutions to two-point boundary value problems for elliptic differential operators.
\end{itemize}

\subsection {Multiresolution Analysis and Wavelets Reference Properties}
Definiion:  A multi-resolution analysis is a decomposition of the Hilbert space $L^2(R^d)$,  $d\ge 1$, into a chain of close subspaces.

$...\subset V_2 \subset V_1 \subset V_0 \subset V_{-1} \subset V_{-2} \subset ....$

such that
\begin{enumerate}
\item $\bigcap _{j\in Z} V_j = \{0\} $
\item For any $f\in L^2 (R^d)$  and $j\in Z$ , $f(x)\in V_j$ if and only if $f(2x)\in V_{j-1}$
\item For any $ f\in L^2 (R^d)$ and any $k\in Z^d$ , $f(x)\in V_0$ if and only if $f(x-k)\in V_0$ 
\item There exist a function $ \phi \in V_0$ such that $\phi (x-k)$ is an orthogonal basis at $V_0$.
\end{enumerate}

Let $W_j$ be an orthogonal complement of $V_j$ in $V_(j-1)$ 

$V_j-1 =V_j \oplus W_j$

and represent $ L^2(R^d)= \oplus W_j $ (as a direct sum). 


\subsubsection {Consequences of the Multi-resolution Definition}

\begin{enumerate}
\item The function may be described as a linear combination of the basis function.
\item Second orthogonality implies the multiplicand (filter) unit, in the power domain, and does not add or remove power from the original.
\end{enumerate}

There are several mathematical transforms to illustrate these points 
\begin{itemize}
\item $\phi$ forms an orthonormal basis for V $(\phi(x-k)) \forall k \in Z$ .
\item $\psi$ forms an orthonormal basis for W $(\psi (x-k)) \forall k \in Z$
\end{itemize}
 
Condition for exact reconstruction for a pair of the quadrature mirror filters:


It may be a good idea to work out this proof a show the results.  

The coefficents of quadrature mirror filters H and G are computed by solving a set of algebraic equations.  

Once the filter H has been chosen, it completely determines the functions $\phi$  and $\psi$ and furthermore multi-resolution analysis.  Furthermore, the QMF are perform manipulation on the original signed to acquire the transform results.  

There is a non-standard form.

In $L^2(R^2)$ the supports of the basis functions are rectangles of various dyadic sizes.  Construction may be done scaling and wavelet functions.  

\subsection{Introduction of the non-standard form:}


Given

 $T_j=
\left(
\begin{array}{ccc}
%  &   &   \\
%  &   &   \\
{A}_{j+1}  & B _{j+1} \\
\Gamma _{j+1} & T_{j+1} \\
%  &   &   
\end{array}
\right)
$

Let $\alpha _{i,l}$ , $\beta _{i,l}$ , $\gamma _{i,l}$, and $\tau_{i,l}$ represent the individual elements of A, B, $\Gamma$, T.  The matrices have the following mapping:
\begin{itemize}
\item $A_j:W_j \rightarrow W_j$
\item $B_j:V_j \rightarrow W_j$
\item $\Gamma _j: W_j \rightarrow V_j$
\end{itemize}

For each element, there are set of equations that describe this mapping.
\begin{itemize}
\item $\alpha _{k,k'} = \int \int K(x,y) \psi _{j,k} (x) \psi _{j,k'} (y) dxdy $
\item $\beta _{k,k'} = \int \int K(x,y) \phi _{j,k'} (y) \psi _{j,k} (x) dxdy $
\item $\gamma _{k,k'} = \int \int K(x,y) \phi _{j,k} (x) \psi _{j,k'} (y) dxdy $
\item $\tau _(k,k') = \int \int K(x,y) \phi _{j,k} (x) \phi _{j,k'} (y) dxdy $
\end{itemize}

\subsection {Standard Form}
Use of the T operator 
\begin{itemize}
\item Calderon-Zygmund
\item pseudo-differential operators
\end{itemize}

Two ways of computing the standard form of a matrix 
\begin{itemize}
\item Non-standard form follows by column-row fix
\item Modified Vector-Matrix Method of Wavelet Transform (Pivot Vector-Matrix Wavelet Transform)
\end{itemize}

\subsection {Compression Operator}
Hypothesis: Operator T is a Calderon-Zygmun or a pseudo-differential operator then by using the wavelet basis with M vanishing moments, which forces the operators $ 
 \left\{ 
 {A_j , B_j, \Gamma _j  } 
 \right\} 
  _{j\in Z } $
to decay roughly as $\frac{1}{d^{n+1}}$ where d is the distance from the diagonal.

Example: Let the kernel satisfy the conditions
\begin{itemize}
\item $|\kappa (x,y)| < \frac{1}{|x-y|} $
\item $|\partial ^M _x \kappa (x,y)| + |\partial ^M _y \kappa (x,y)| < \frac{c_0}{|x-y|^{1+M}}$
\end{itemize}

Furthermore, the choice of wavelet basis with M vanishing moments (such that $M \ge 1$) 
yields the following condition with the non-standard form coefficients:

$|\alpha ^j _{i,l}| + |\beta^j _{i,l}| + |\gamma^j _{i,l}| \le
 \frac{c_m}{1+|i-l|^{m+1}}$

$\forall (|i-l|\ge 2M)$

If $  | \int_{I\times I} \kappa (x,y) dxdy | \le C|I| \forall dydx    $
intervals then $\forall i,l \in Z$ 


Hypothesis on Pseudo-differential operator
If T is a pseudo-differential operator with symbol $\sigma (x, \xi )$ of order $\lambda$ define by the formula 

$ T(f)(x) = \sigma(x,D)f = \int e^{ix} \xi \sigma (x,\zeta) \hat{f}(\zeta) d\xi $

$T(f)(x) = \int \kappa (x,y) f(y)dy $

where $\kappa$ is the distributed kernel of T, then assuming that the symbols $\sigma$ of T and $\tilde{\sigma}$ of $\tilde{T}$ satisfy the standard conditions:

$|\partial ^\alpha _\xi \partial ^\beta _x \sigma (x,\xi)| \le C_{\alpha , \beta} (1+|\xi |)^{\lambda - \alpha + \beta}$

$ | \partial ^\alpha _\xi \partial ^\beta _x \hat{\sigma} ( x , \xi ) | \le C_{\alpha , \beta} (1 + |\xi | )^{\lambda -\alpha + \beta}  $

The following inequality holds $\forall i,l$:

$ |\alpha ^j _{i,l} | + | \beta ^j _{i,l} | +  |\gamma ^j _{i,l}| \le \frac {2^{\lambda j}  C_m } {1 + |i-l| ^{m+1}} $
 
 
Key points:
\begin{itemize}
\item The operator-compression hypothesis are used to both so a reduction of a matrix to diagonal form for use of linear algebra techniques be applied to these problems.  
\item Also these hypothesis are used provide support for G. David and J.L. Jorne Theorem on compression of operators to yield fast algorithms.
\item An vector matrix multiply is used to show an example of this theorem in practice.  
\end{itemize}


\subsection {The differential operator in wavelet bases.}

Theme: For a number of operators we may compute the non-standard form in the wavelet bases by solving a small system of linear equations.  

Example: Non-standard form of the operator:

Compute $\alpha^j _{i,l}$ , $\beta^j _{i,l}$ and $\gamma^j _{i,l}$ of $A_j$, $B_j$, and $\Gamma_j$ where $i,l \in Z$ for the operator $\frac{d}{dx}$.  
\begin{itemize}
\item $\alpha ^j_{i,l} = 2^{-j} \int \psi (2^{-j} - i) \psi\prime (2^{-j}x -l) 2^-j dx = 2^{-j} \alpha_{i-l}  $
\item $\beta ^j_{i,l} = 2^{-j} \int \psi (2^{-j} - i) \phi\prime (2^{-j}x -l) 2^-j dx = 2^{-j} \beta_{i-l}  $
\item $\gamma ^j_{i,l} = 2^{-j} \int \phi (2^{-j} - i) \psi\prime (2^{-j}x -l) 2^-j dx = 2^{-j} \gamma_{i-l}  $
\end{itemize}

where 
\begin{itemize}
\item $\alpha_l = \int\limits ^\infty _{-\infty} \psi (x-l)\frac{d}{dx} \psi (x)dx$
\item $\beta_l = \int\limits ^\infty _{-\infty} \psi (x-l)\frac{d}{dx} \phi (x)dx$
\item $\gamma_l = \int\limits ^\infty _{-\infty} \phi (x-l)\frac{d}{dx} \psi (x)dx$
\end{itemize}

For example using 
\begin{itemize}
\item $\phi(x) = \sqrt{2} \sum\limits^{k-1} _{k=0} h_k \phi(2x-k)$

\item $\psi(x) = \sqrt{2} \sum\limits^{k-1} _ {k=0} g_k \phi (2x-k)$

\item $\alpha_i = 2 \sum\limits_k \sum\limits_{k\prime} g_k g_{k\prime} r{2i + k -k\prime}$

\item $\beta_i = 2 \sum\limits_k \sum\limits_{k\prime} g_k h_{k\prime} r{2i + k -k\prime}$

\item $\gamma_i = 2 \sum\limits_k \sum\limits_{k\prime} h_k g_{k\prime} r{2i + k -k\prime}$
\end{itemize}

$r_l = \int\limits^\infty _{-infty} \phi(x-l) \frac{d}{dx}\phi(x)dx$

Conclusions, eye catchers, and questions
\begin{enumerate}
\item The representation of $d/dx$ is completely determined by the coefficents $r_l$ or more to the point by the representation on the subspace $V_0$.
\item The coefficents $r_l$ depend only on the auto-correlation function of the scaling functions $\phi$, rather than the scaling function itself.  
\item Justification of one of the conclusions (2) is that integral depends on $|\phi(\xi)|^2$.  
\item Also, how does Beylkin get to this point?
\end{enumerate}


Proposition: The goal is to reduce the computation of the coefficients $r_l$ to solving a system of linear algebraic equations.

Given:

$r_l = \int\limits^\infty _{-\infty} \phi (x-l) \frac{d}{dx} \phi(x)dx $

$ r_l = \int\limits^\infty_{-\infty} |\phi (\xi)|^2 (i\xi)e^{-il\xi} d\xi   $

 Proposition 1: If the given exists, then the following coefficents $r_l , l \in Z$ satisfy the following system of linear algebraic equations:
 
$r_l = 2( r_{2l} +\frac{1}{2} \sum\limits ^{L/2}_{k=1} \alpha_{2k-1} (r_{2l-2k+1} + r_{2l+2k-1}))$

$\sum\limits_l lr_l =-1$

such that $a_{2k-1} = 2 \sum\limits^{L-2k}_i=0 h_ih_{i+2k-1} $   $k\in U[1,L/2]$.  are the auto-correlation coefficients of the filter H.

If $M \ge 2$, then a and b have an unique solution with a finite of non-zero $r_l$, namely $r_l\not=0 \forall l\in [-L+2,L-2]$ and $r_l = -r_l$

% Reference Data Analysis for High Energy Physics 2nd Edition by R. Fruhwirth, M. Regler, R. K. Bock, H. Grote, D. Notz, c. 2000 Publisher: Cambridge University Press ISBN: 0-521-63219-6 hardback :  
% Useful for pattern recognition and Kalman filters.

%Reference:  Astronomical Data Analysis II:  Volume 4847 Procedings of SPIE: The International Society for Optical Engineering.

% Reference:  Tools to detect non-Gaussianity in non-standard cosmological models by J.E. Gallegos, Martinez-Gonzalez, F. Argueso, L. Cayon, and J.L. Sanz of Instituto de Fisica de Cantabria E39005; Santander, Spain Universidad de Oviedo E33007 Oviedo, Spain.




 \end{document}