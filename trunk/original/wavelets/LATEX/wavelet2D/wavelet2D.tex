
\documentclass{article}
%%%%%%%%%%%%%%%%%%%%%%%%%%%%%%%%%%%%%%%%%%%%%%%%%%%%%%%%%%%%%%%%%%%%%%%%%%%%%%%%%%%%%%%%%%%%%%%%%%%%%%%%%%%%%%%%%%%%%%%%%%%%
\usepackage{graphicx}
\usepackage{amsmath}

%TCIDATA{OutputFilter=LATEX.DLL}
%TCIDATA{Created=Thu Feb 20 14:24:52 2003}
%TCIDATA{LastRevised=Thu Feb 20 23:21:05 2003}
%TCIDATA{<META NAME="GraphicsSave" CONTENT="32">}
%TCIDATA{<META NAME="DocumentShell" CONTENT="General\Blank Document">}
%TCIDATA{CSTFile=LaTeX article (bright).cst}

\newtheorem{theorem}{Theorem}
\newtheorem{acknowledgement}[theorem]{Acknowledgement}
\newtheorem{algorithm}[theorem]{Algorithm}
\newtheorem{axiom}[theorem]{Axiom}
\newtheorem{case}[theorem]{Case}
\newtheorem{claim}[theorem]{Claim}
\newtheorem{conclusion}[theorem]{Conclusion}
\newtheorem{condition}[theorem]{Condition}
\newtheorem{conjecture}[theorem]{Conjecture}
\newtheorem{corollary}[theorem]{Corollary}
\newtheorem{criterion}[theorem]{Criterion}
\newtheorem{definition}[theorem]{Definition}
\newtheorem{example}[theorem]{Example}
\newtheorem{exercise}[theorem]{Exercise}
\newtheorem{lemma}[theorem]{Lemma}
\newtheorem{notation}[theorem]{Notation}
\newtheorem{problem}[theorem]{Problem}
\newtheorem{proposition}[theorem]{Proposition}
\newtheorem{remark}[theorem]{Remark}
\newtheorem{solution}[theorem]{Solution}
\newtheorem{summary}[theorem]{Summary}
\newenvironment{proof}[1][Proof]{\textbf{#1.} }{\ \rule{0.5em}{0.5em}}
%\input{tcilatex}

\begin{document}


\section{Class 2-D Wavelet: Complete Form}

\bigskip Many concepts are evident in the convolution method wavelet
transforms. \ Why? \ It is true that a simple matrix multiply can also
provide means of transformation. \ However, a convolution provides a simpler
concept. \ One goal of these experiments is to measure computational
strength of each method. \ 

\bigskip A transform must take a given matrix, and produce a result which
contains four components: average, vertical, horizontal and diagonal
difference. \ The can be done by either over complete or complete one
dimensional wavelet transform means. \ 

\bigskip The over complete is addressed in the next class. \ A complete
transform method returns a result matrix which is the same size as the
source matrix. \ The result contains the four compoents. \ Each compoent
resides on 4 corners of the matrix. \ Given a matrix B, the transform is to
yield the following form:

\qquad $B\Rightarrow 
\begin{array}{cc}
H & D \\ 
A & V
\end{array}
$

\qquad where A is the average matrix, H is the horizontal matrix, V is the
vertical compoent, and D is the Difference Matrix. \ \ \ Why could this
transform have the form:

$\qquad B\Rightarrow 
\begin{tabular}{ll}
$A$ & $V$ \\ 
$H$ & $D$%
\end{tabular}
$ 

\bigskip In truth, it is not obvious. \ In the first transform has the A
compoent as the row and column average. \ So it is in the second. \ In both
cases, V is the result of the difference row wise, and average column wise.
\ Like wise, H is the average of the rows, and difference of columns. \
Finally, D is the difference of rows and difference of the columns. \ \
Furthermore, each case involve piping the results of row operation into the
column operations or vise-versa. \ 

\bigskip This set of classes uses the first. \ Only reason for choosing such
a method at the time of this classes construction is convention. \ \ Such a
convention allows a standard comparision for accuracy, on a known solution.
\ Further classes are to explore results via non-conventional means. \ Also,
it is hoped that properties of each method will be revealed. \ \ \ 

This class provides three functions. \ The column wavelet provides a
transform solely on the columns. \ The row wavelet likewise, provides a
wavelet transform on each of the rows. \ The 2-D\ wavelet transform is
simple a row wavelet followed by a column wavelet transform. \ 

It should be noted that these wavelet transforms are Haar Wavelet
transforms.. Multiple resolutions functions can be made, but again it is
still the Haar Wavelet at the core. \ Other class clones may be produced to
use other wavelet basis such as Daubachie or Coeflet. \ 

Another support class required for support of the two dimensional wavelet is
the image reader/ writer. \ Such a class of functions provide means of
acquiring data for a matrix to be computed. \ Granted this matrix could have
been populated by other means. \ Generating such a class consumed some time
on this project. \ The reason is that there a few mechanisms for doing this.
\ One is to use raw image types. \ These types are known as by their
extensions: pgm and ppm. \ Another popular mechanism is Image Magick. \ The
reason for Image Magick is that it is available on most UNIX platforms
(Linux, IRIX, Solaris, and Mac OSX). \ Otherwise, schemes such as Apple's
Quicktime would be used. \ The other reason for the use of other software to
acquire the image is that the objective of this class is not produce image
translators for each image type imaginable. \  

\subsection{\protect\bigskip Method: Column Wavelet Transform}

The column wavelet transform is provides a source matrix, and returns a
result. \ In order to yield this result, each column is extracted into a
vector and that vector is fed into a one-dimensional transform. \ The result
of the one dimensional transform is placed the corresponding row of the
result matrix. \ Mathematically, this algorithm is as follows:

\qquad $\forall j\in col$

\qquad \qquad n = 0

\qquad \qquad $\forall i\in row$ in reverse order

\qquad \qquad \qquad $S_{n++}=$source$_{i,j}$

\qquad \qquad $S$ $\overset{W}{\Rightarrow }R$

\qquad \qquad n = 0

\qquad \qquad $\forall i\in row$ in reverse order

\qquad \qquad \qquad result$_{i,j}$ \ = $R_{n++}$

\bigskip Note that the need for the n index is keep the order straight for
this wavelets convention. \ 

\bigskip 

\subsection{Method: Row Wavelet Transform}

\bigskip Like the column wavelet transform, the row wavelet transform takes
a source matrix and returns a resulting matrix. \ The only two significant
differences are first the items being operated on (rows not columns). \
Second, the lack of need for the n index. \ 

\qquad $\forall i\in row$

\qquad \qquad $\forall j\in col$ 

\qquad \qquad \qquad $S_{j}=$source$_{i,j}$

\qquad \qquad $S$ $\overset{W}{\Rightarrow }R$

\qquad \qquad n = 0

\qquad \qquad $\forall j\in col$ in reverse order

\qquad \qquad \qquad result$_{i,j}$ \ = $R_{j}$

\subsection{Method:Wavelet Transform}

\bigskip As stated above, the two dimensional wavelet transform is simply
row wavelet transform followed by a column wavelet transform. \ It could
have been done in reverse order. \ However, the result would be the same. \ 

Just of note, this class lacks for the moment an inverse transform function.
\ Not that such a thing does not exist mathematically. \ It simply was not
implemented at the time of this document. \ 

\bigskip 

\bigskip 

\section{Class 2-D Wavelet Over Complete}

The point of the over complete wavelet transform is to preserve the
mathematical properties that are produced by the transform itself. \ Since
the original signal can be recovered from the complete wavelet transform,
either method can be used without energy loss. \ \ Also the over complete
version takes up more space the original. \ 

\bigskip Once again, the reason for the over complete is to preserve the
mathematical properties of the transform. \ Also, the over complete form
allows for individual compoents to be computed without bothering with the
remaining components being computed. \ 

The functions included in the Over Complete 2-D Wavelet Transform class
include computing the row average, row difference, column average and column
difference. \ Also include are functions to compute the four 2-D wavelet
matrix components (averages, vertical, horizontal, and diagonal difference).
\ Some of these functions are simply wrappers around other class functions.
\ 

\section{April 5, 2003}
\subsection{Multi-Resolution 2-D Wavelet Transform}
The algorithm is as follows:

$\forall s \in resolution $
\begin {enumerate}
\item Let k = $\frac{rows}{2^s}$
\item Let l = $\frac{columns}{2^s}$
\item $\forall i \in k $ Row Transform on ith row up to column l.
\item $\forall j \in l $ Column Transform on the jth column up to the kth row.
\end {enumerate}

The inverse transform is as follows:

$\forall s \in resolution $ in reverse order
\begin {enumerate}
\item Let k = $\frac{rows}{2^s}$
\item Let l = $\frac{columns}{2^s}$
\item $\forall i \in k $ Column Inverse Transform on jth column up to the kth row.
\item $\forall j \in l $ Column Transform on the ith row up to the lth column.
\end {enumerate}

Sounds simple.  What is needed for all of the xform and inverse xform calls to work are:
\begin{enumerate}
\item A Filter for Haar average
\item A filter for Haar difference
\item source matrix
\item result matrix
\end{enumerate}

Inverse Transform components:

Column Inverse Xform:

\begin{enumerate}
\item let k = row
\item let k2 = $\frac{row}{2}$
\item $\forall i \in [0,k2) $
\begin{itemize}
\item $sY[2i]=(T[i][j]-T[k2+i][j] ) \sqrt{1/2}$
\item $sY[2i+1]=(T[i][j]+T[k2+i][j] ) \sqrt{1/2}$

\end{itemize}
\item $\forall i \in [0,k) T[i][j] = sY[i]$
\end{enumerate}

Row Inverse Xform:

\begin{enumerate}
\item let k = column
\item let k2 = $\frac{column}{2}$
\item $\forall j \in [0,k2) $
\begin{itemize}
\item $sX[2i]=(T[i][j]-T[i][k2+j] ) \sqrt{1/2}$
\item $sX[2i+1]=(T[i][j]+T[i][k2+i] ) \sqrt{1/2}$

\end{itemize}
\item $\forall i \in [0,k) T[i][j] = sY[i]$
\end{enumerate}

Note: sY and sX belong to the class and are allocated and deallocated in the wavelet inverse xform method. 

Row X Form 

\begin{enumerate}
\item s = column
\item s2 = column/2
\item $\forall k \in [0,s) xA[k] = xD[k] = 0.0$
\item $\forall k \in [0,s) \forall l \in [0,l)$
\begin {itemize}
\item n = k -l
\item if ( $n \in [0,s)$) 
\begin{itemize}
\item $xA[k] += W[i][n] * hA[l]$
\item $xD[k] += W[i][n] * hD[l]$
\end {itemize}
\end {itemize}
\item $\forall k \in [0,s2) $
\begin{itemize}
\item $W[i][k] = xA[2k+1]$
\item $W[i][k+s2] = xD[2k+1]$
\end{itemize}
\end{enumerate}

Column X Form 

\begin{enumerate}
\item s = row
\item s2 = row/2
\item $\forall k \in [0,s) yA[k] = yD[k] = 0.0$
\item $\forall k \in [0,s) \forall l \in [0,l)$
\begin {itemize}
\item n = k -l
\item if ( $n \in [0,s)$) 
\begin{itemize}
\item $yA[k] += W[n][j] * hA[l]$
\item $yD[k] += W[n][j] * hD[l]$
\end {itemize}
\end {itemize}
\item $\forall k \in [0,s2) $
\begin{itemize}
\item $W[k][j] = xA[2k+1]$
\item $W[k+s2][j] = xD[2k+1]$
\end{itemize}
\end{enumerate}

The multi-resolution wavelet transform and inverse multi-resolution transform resembles the vector-matrix version.  Convolution is built into the transform.  However, there are structural changes.  

The wavelet transform (multiresolution) uses private members of the class (hA, hD, xD/yD, xA/yA).  Both Haar filters are maintained this way.  Also both row and column transforms have average and difference myVector classes for temporary storage.   All of these members are allocated and destroyed from the wavelet transform method itself.   The simplified algorithm of the row transform is:
\begin{enumerate}
\item initialize xA and xD to zero
\item $\forall k \in columns \forall l \in filter$
\begin{itemize}
\item n = k - l
\item if ( $n \in columns $ )

	$xA_k = W_{i,n} * hA_l $
	
	$xD_k = W_{i,n} * hD_l $
	
	
\end{itemize}
\item Transfer back to W

	$W_i = xA|xD $
\end{enumerate}

\begin{enumerate}
\item initialize yA and yD to zero
\item $\forall k \in rows \forall l \in filter$
\begin{itemize}
\item n = k - l
\item if ( $n \in columns $ )

	$yA_k = W_{i,n} * hA_l $
	
	$yD_k = W_{i,n} * hD_l $
	
	
\end{itemize}
\item Transfer back to W

	$W_j = yA|yD $
\end{enumerate}

Note: $W_i$ are the row vectors and $W_j$ are the column vectors, and $W_{i,j}$ is the element from the ith row and jth column.

\subsubsection {Computational Cost:}
The cost of this algorithm is computed by the per row and per column cost, and then for the matrix and then for the multi-resolution steps.   Per row the cost is 3k, where k is the number of columns.  Per column the cost is 3l, where l is the number rows.  For the whole matrix, one resolution costs 6kl operations to compute the wavelet transform.    Per resolution, the rows and columns shrink by $2^i $ for each resolution, i, performed.  The limit of this cost comes up to $12kl$ operations.  Thus the cost is linear.  

\subsection {Results}
The expected result is a picture within a picture.  Each average component has a further transform on it.  The three resolution transform has the form:

$W_3 =$ 
\begin{tabular}{ll}
\begin{tabular}{ll}
\begin{tabular}{ll}
$A_3$& $V_3$ \\ 
$H_3$ & $D_3$%
\end{tabular} $V_2$ \\ 
$H_2$ & $D_2$%
\end{tabular} & $V_1$ \\ 
$H_1$ & $D_1$%
\end{tabular}

Refer to Figure for the image transform results.  

To obtain the inverse, an exact reverse procedure is necessary, otherwise the distortion is hideous.   The first attempt of the wavelet inverse transform was out of order, refer to Figure .  A correct picture was obtained during the second attempt.  Correct order yielded correct results, refer to Figure .

\end{document}
