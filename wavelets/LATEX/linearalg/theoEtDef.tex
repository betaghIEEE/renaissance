\documentclass[11pt]{article}
\usepackage{graphicx}
\usepackage{amssymb}
\usepackage{epstopdf}
\DeclareGraphicsRule{.tif}{png}{.png}{`convert #1 `basename #1 .tif`.png}

\textwidth = 6.5 in
\textheight = 9 in
\oddsidemargin = 0.0 in
\evensidemargin = 0.0 in
\topmargin = 0.0 in
\headheight = 0.0 in
\headsep = 0.0 in
\parskip = 0.2in
\parindent = 0.0in

\newtheorem{theorem}{Theorem}
\newtheorem{corollary}[theorem]{Corollary}
\newtheorem{definition}{Definition}

\title{Good old Linear Algebra and Theorems}
\author{The Author}
\begin{document}
\maketitle

Definition: Let F be a field.  A vector space V over F is  set with operations of vector addition + and scalar $\dot$ satisfying the following properties:
\begin{enumerate}
\item Closure for addition:  $\forall u,v\in V$, $u+v$ is defines and is an element of V.
\item Communitive of addition $u+v=v+u$  $\forall u,v \in V$
\item Associativity for addition: $u + (v +w) = (u+v)+w$  $\forall u,v,w\in V$
\item Existence of additive identity $\forall u\in V$ : $u+0 = u$
\item Existence of Additive Inverse $\forall u \in V$ , there exists an element in V, -u,  such that $u + (-u) = 0$.
\item Closure for scalar multiplication: $\forall \alpha \in F$ and $\forall u \in V$ $\alpha \dot u$ is well defined.  
\item Behavior of the scalar multiplication identity: $\forall u \in V$ $\exist \alpha$ called 1 
\item Associativity fir scalar multiplication:$\forall u \in V$ $\alpha \dot (\beta \dot u) = (\alpha \dot \beta) \dot u$
\item Distribution properties: $\forall \alpha , \beta \in F$ and $u,v \in V$ 
	$\alpha \dot (u + v) = \alpha u + \alpha v$
	$(\alpha + \beta) \dot u = \alpha \dot u + \beta \dot u$
\end{enumerate}


\begin{thebibliography}{99}
\bibitem Michael W. Frazier \textsl{An Introduction to Wavelets through Linear Algebra} 1999 Springer Verlag New York, NY 10010, Inc.
\end{thebibliography}
 \end{document}