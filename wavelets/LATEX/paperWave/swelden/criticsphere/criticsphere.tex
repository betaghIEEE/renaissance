\documentclass[11pt]{article}
\usepackage{graphicx}
\usepackage{amssymb}
\usepackage{doublespace}

\textwidth = 6.5 in
\textheight = 9 in
\oddsidemargin = 0.0 in
\evensidemargin = 0.0 in
\topmargin = 0.0 in
\headheight = 0.0 in
\headsep = 0.0 in
\parskip = 0.2in
\parindent = 0.0in

\newtheorem{theorem}{Theorem}
\newtheorem{corollary}[theorem]{Corollary}
\newtheorem{definition}{Definition}

\title{Critic of Spherical Wavelets}
\author{Daniel D. Beatty}
\begin{document}
\maketitle
%Reference Spherical Wavelets: Efficiently Representing Functions of the Sphere by Peter Schroder and Wim Sweldens.

Abstract Ideas The Power of the Wavelet

\begin{itemize}
\item Their power lies in the fact that they only require a small number of coefficents to represent general functions and large data sets accurately.
\item Scalar functions defined on the sphere using biorthogonal wavelets
\item Custom properties 
\item lifting scheme
\item Reparameterizing directions over the set of visible surfaces
\item Mapping directions to the unit square providing the avenue for construction of spherical wavelet.
\end{itemize}

The bases claim to allow fully adaptive subdivisions.

Examples
\begin{itemize}
\item Topographical Data
\item Bi-direction reflection distribution functions
\item Illumination techniques
\item ``The offer both theoretical characterization of smoothness, insights into the structure of functions and operators, and practical numerical tools which lead to faster computational algortihms.''
\item The challenge is to construct wavelets in general domains as they appear in graphics applications.  
\end{itemize}

Reference to Lounsbery et al (LDW) for 
\begin{itemize}
\item representing surface themselves
\item representing functions defined on a surface.
\end{itemize}

Reference: Dahlke et al.  intro to spherical wavelet (first) used a tensor product basis where one factor is an exponential spline.  

Practical Applications:
\begin{enumerate}
\item Computer Graphics
\begin{enumerate}
\item Manipulations of spherical objects
\item display of spherical objects
\item remote sensing imagery
\item simulation and modeling of bidirectional reflection distribution functions.
\item illumination algorithms
\item direction information processing (env maps and sphere views)
\end{enumerate}

\end{enumerate}

Outline of paper (in general) 
\begin{enumerate}
\item Brief review of applications (computer graphics and others).
\item Wavelets on the sphere 
\item Basic machinery of lifting and the fast wavelet x-form
\item Implementation
\item Simulations 
\end{enumerate}

Representing Functions on a sphere 
\begin{itemize}
\item Dalton and geodesic sphere
\item Hierarchical and non-hierarchical forms
\item Sparseness is key
\end{itemize}

Example: Bidirectional reflectance distribution functions and radiance
\begin{itemize}
\item BRDF : $f_r (\vec{w_i} , \vec{x}, \vec{w_o})$ describes the relationship between 
\begin{itemize}
\item $\vec{w_i}$ incoming radiance
\item $\vec{x}$ point 
\item $\vec{w_0}$ outgoing radiance
\end{itemize}

\item Spherical Harmonics needed to describe BRDF (involves the Fourier domain).
\item A FFT has not been extended to the sphere domain
\item One method derived from Monte Carlo simulations of micro-geometry.
\end{itemize}

Example: Radiance $L(\vec{x}, \vec{w})$ function.
\begin{itemize}
\item Defined over all surfaces and directions
\item Global support introduces errors and ringing
\end{itemize}

Trouble for the spherical wavelet
\begin{itemize}
\item Suffers from distortions
\item Contaminated wavelets from parameterization 
\end{itemize}

Wavelets on the sphere (2nd generation wavelets)
\begin{itemize}
\item Note local support for compact support
\item Smoothness (decay toward high-frequencies)
\end{itemize}

Base Philosophy:  
Build wavelets with all desirable properties (localization, fast-transfer ) adapted to much more general settings than the real line.

One difference with classical wavelets are not the same throughout, but change locally to reflect the changing nature of the surface and its measure.

``In the setting of second generation wavelets, translation and dialation can no longer be used, and the Fourier transform becomes worthless as a construction tool.''

Multi-resolution:  Given A function $L_2 = L_2 (S^2, d\omega)$.  Call functions of finite energy defined over $S^2$.  

Multi-resolution is defined as a sequence of closed subspaces $V_j \subset L_2$ with $j\ge 0$ such that 
\begin{itemize}
\item $V_j \subset V_{j+1}$
\item $\cup_{j\ge0} V_j $ is dense in $L_2$
\end{itemize}

$K (j)$ is a general index set.

Note that every scaling function $\phi_{j,k}$, there exists coefficents $\{h_{i,j,k}\}$ such that

$\phi_{j,k} = \sum\limits_l h_{j,k,l} \phi_{j+1,l}$
\begin{itemize}
\item $j \ge 0 $
\item $k \in K(j)$
\item $l \in K(j+1) $
\item $h_{j,k,l} = h_{l-2k}$ in the classic wavelet sense
\end{itemize}

Also, there is a concept of nested spaces, $V_j$, with bases of $\tilde{\psi}_{j,k}$ (dual-scan basis functions ) which are bi-orthogonal to the scaling functions.  

If the scaling functions and dual scaling functions coincide, then the scaling function forms an orthogonal basis.  

If the multi-resolution and dual-multi-resolution analysis coincide, (not necessarily the scaling and dual scaling functions) then the scaling functions are semi-orthogonal.   

Swelding's opinion on orthogonality
\begin{itemize}
\item Global support 
\item Assumption that coincidence does not occur
\item General case is for bi-orthogonal cases 
\end{itemize}

Wavelet Constructions: A basis space $w_j$ that encodes the difference between two successive levels representation.  Conditions for the wavelet basis space are as follows.  

Wavelets: Multi-resolution wavelet construction 

\begin{enumerate}
\item $V_j \oplus W_j$
\item The set of basis functions $\{\psi-{j,m} | j\ge , m \in M(j)\}$ such that $M(j)\subset X(j+1)$
\begin{enumerate}
\item is a Riesz basis for $L_2(S^2)$
\item is a Riesz basis of $W_j$
\end{enumerate}

\item $\psi_{j,m}$ defines a spherical wavelet basis
\item $\psi_{j,m} = \sum_l g_{j,m,l} \phi_{j+1,l}$
\item Wavelets $\psi_{j,m}$ have $\bar{N}$ vanishing moments s.t. 
\begin{enumerate}
\item for $0\le i \le \bar{N} \exists <\psi{j,m}, P_i> = 0 $
\item $ P_i$ are polynomials and $\bar{N}$ is independent of it. 
\end{enumerate}
\item $ P_i$ are defined as the restriction to the sphere.  
\item There are basis and dual basis functions $( \tilde{\psi}_{j,m} )$ which are bi-orthogonal.  
\item The following equalities hold:  
\begin{enumerate}
\item $<\tilde{\psi}_{j,m}, \phi_{j,k}> = <\tilde{\phi}_{j,k}, \psi_{j,m}> = 0$
\item $f= \sum_{j,m} <\tilde{\psi}_{j,m} , f> \psi_{i,m} = \sum_{j,m} \gamma _{j,m} \psi _{j,m}$
\end{enumerate}

\end{enumerate}


A Reisz base has some Hilbert space is a countable suset $\{f_k\}$ so that every element f of the space can be written uniquely as $f_i =\sum_k c_k f_k$ and positive constants A and B exist with 

$A||f||^2 \le \sum_k |c_k|^2 \le B||f||^2 $

The conditions for constructing a wavelet scaling function are simply function of coarser scaling functions.

$\psi_{j+1,l} = \sum\limits_k \tilde{h} \phi_{j,k} = \sum\limits_m \bar{g}_{j,m,l} \psi_{j,m}$

such that 
\begin{itemize}
\item $k\in K(j)$
\item $l\in K(j+1)$
\item $m\in M(j)$
\end{itemize}

Proposition:  
Given a set of scaling function coefficients of function f.

$\{ \lambda_{n,k} = <f,\emptyset> | k\in K(x) \}   $

such that n is some finest resolution level.  

Results:
The fast wavelet transform recursively calculates the coarser approximation to the underlying functions.  
\begin{itemize}
\item $ \{\gamma_{j,n} | 0 \le j < n; m \in M(j) \} $
\item $\{ \lambda_{0,k} | k\in (\emptyset) \}$
\end{itemize}

One step in the FWT computes a coarser level from a finer level $( j+1 \to j)$
\begin{itemize}
\item $\lambda_{j,k} = \sum\limits _l \tilde{h}_{j+1,l} \lambda_{j+1,l}$
\item $\gamma_{j,m} = \sum\limits_l \tilde{g}_{j,m,l} \lambda_{j+1,l} $
\end{itemize}

A single step in the individual transform that computes the finer level from the coarser level  $(j+1 \leftarrow j)$. 


$\lambda_{j+1,l} = \sum\limits_k h_{j,k,l} \lambda_{j,k} + \sum\limits_m g_{j,m,l} \gamma_{j,m} $

 \end{document}