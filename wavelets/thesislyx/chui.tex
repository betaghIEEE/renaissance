\documentclass[11pt]{article}
\usepackage{graphicx}
\usepackage{amssymb}
\usepackage{epstopdf}
\DeclareGraphicsRule{.tif}{png}{.png}{`convert #1 `dirname #1`/`basename #1 .tif`.png}

\textwidth = 6.5 in
\textheight = 9 in
\oddsidemargin = 0.0 in
\evensidemargin = 0.0 in
\topmargin = 0.0 in
\headheight = 0.0 in
\headsep = 0.0 in
\parskip = 0.2in
\parindent = 0.0in

\newtheorem{theorem}{Theorem}
\newtheorem{corollary}[theorem]{Corollary}
\newtheorem{definition}{Definition}

\title{Brief Article}
\author{The Author}
\begin{document}
\maketitle


\section {Wavelet Theory}

Qualities/ behavior of spectral information acquired by Fourier Transform:
\begin{itemize}
\item Full knowledge must be known about signal in the time domain
\item Small changes in small neighborhoods in the time domain drastically affect the frequency domain drastically affects the frequency domain
\item Real Time Analysis of non-stationary is not practical with Fourier Transform
\item Improvements are made with Gabor windowing function, short\_time Fourier Transform, and the integral wavelet transform 
\item Things to watch out for
\begin{itemize}
\item Uncertainty principle as it applies to the window size
\item Scaling and dilation of the basis function
\end{itemize}

\end{itemize}
 
\subsection {The Gabor Transform}
All analog signals are represented in function, $f\in L^2(R)$

The Fourier Transform 
\[\hat{f}(\omega) = \int ^\infty _{-infty} e^{-i \omega t} f(t)dt \]

such that 
\begin{itemize}
\item $t = $ time 
\item $\omega = $ frequency
\end{itemize}

Given Gaussian function 
\[ g_a (t) = \frac{1}{2\sqrt{\pi a}} e^{-\frac{t^2}{4\alpha}} \]

The Gabor Transform 
\[ (G^\alpha _b f) (\omega) = \int ^\infty _{-\infty } (e^{-i\omega t} f(t)) g_\alpha (t-b)dt \]

The purpose of this transform is to focus the Fourier Transform around a particular segment (a window) of signal.

The RMS or standard deviation determines the scaling factors, and the width of the Gabor Transform.
\subsubsection {Interpretation of the Gabor Transform}
Set \[G^\alpha \_{b,\omega} (t) = e^{i\omega t} g_\alpha (t-b)\] 
The Gabor Transform becomes 
\[ (G^\alpha _b f) (\omega) = \lbrace f, G^\alpha _{b, \omega} \rbrace = \int f(t) \bar{G^\alpha _{b,\omega} (t)} dt \]

The Gabor Transform is a windowing function on f

This is similar to the Integral Wavelet Transform and the Convolution Wavelet Transform.

How does the Parseval Identity apply the Gabor Transform of f?

Try this identity
\[  G^\alpha _{b , \omega } (\n)  e^{-ib(\n -\omega)} e^{-\alpha(\n -\omega)^2}  \]
\[ \lbrace G^\alpha _{b , \omega },  f \rbrace (\omega) = \frac {1}{2\pi} \lbrace \hat{f} , \hat{G} ^\alpha _{b,\omega} \lbrace\]

How is this to be interpreted?  
\begin{enumerate}
\item The windowing Fourier Transform of $f$ with windowing function $g_\alpha$ at $t=b$ agrees with window inverse Fourier Transform of $\hat{f}$ with window function $g_{\alpha / 4}$ at $\n = \omega$
\item Second the identity 
\[\lbrace f, G^\alpha _{b, \omega} \rbrace = \lbrace f, H^\alpha _{b, \omega} \rbrace \]
Shows that the information obtained by investigating an analog signal $f(t) $ at $t=b$ by using the window function $G^\alpha _{b,\omega}$ (Gabor Transform window) is the same information by observing the spectrum $\hat{f} (\n)$ of the signal in a neighborhood of the same frequency $\n = \omega$ by using the window function $H^\alpha _{b,\omega}$ 
\item The product of width of time window and frequency window.
\item The cartesian product
\[ \]  % Come back and insert formula
of the time window and frequency window is called the rectangular time-frequency window.
\begin{enumerate}
\item Width of the time frequency window = $2\sqrt{\alpha}$
\item height of the time frequency window $\frac{1}{\sqrt{\alpha}}$
\end{enumerate}
\item Restrictions on the Gabor Transform is the window is constant and does not handle high and low frequency well.

\end{enumerate}


\subsection {Short Time Fourier Transform and the Uncertainty Principle}

Other functions may also be used as windows functions instead.  Such functions must satisfy 
\[w\in L^2(R) \]
\[t w(t) \in L^2(R) \]
\[  |t|^{1/2} w(t) \in L^2(R)  \]
\[w\in L^1 (R) \]
\begin{itemize}
\item The Fourier Transform does not necessarily satisfy this condition and may not be a frequency window 
\item The Haar Wavelet does not satisfy this condition and can not be used for time frequency location?
\item The B-spline does not satisfy this condition and can not be used for time frequency analysis
\begin{itemize}
\item Both are non-continuous 
\item $N_1 \not\in L^1 (R) $ and $\hat{\psi} \not \in L^1(R)$ which is the failure point
\end{itemize}

\end{itemize}

\subsection {Integral Wavelet Transform}
The STFT provides a rigid time frequency window that is unchanged.  

\[ [\omega ^\ast + \omega - \Delta _{\hat{\omega}}, \omega ^\ast + \omega + \Delta _{\hat{\omega}} ] \]

such that $\omega ^\ast + \omega$ is the center $\Delta _\omega $ is the frequency width (bandwidth)

Relative to some basic wavelet, the IWT provides a flexible time frequency window which automatically narrows when observing high frequencies  and widen when studying low frequencies environments.

Definition: If $\psi  \in L^2(R)$ satisfies the ``admissibility'' condition: 

\[   C_\psi = \int ^\infty _{-\infty} \frac {|\hat{\psi}|^2}{|\psi|} d\omega < \infty \]
then  \psi is called a ``basic wavelet.''  Relative to every basic $\psi$ , the integral wavelet transform (IWT) on $L^2 (R)$ is defined by 
\[ (W_\psi f) (b,a) = |a|^\frac{1}{2} \int ^\infty _{-\infty} f(t) \bar {\psi (\frac{t-b}{a})} dt \]
$f \in L^2 (R) $ where $a,b \in R$ with $a \not= 0$ 

IWT defined 

\[  \psi _{b,a} (t) = |a|^{-\frac{1}{2}} \psi (\frac{t-b}{a} )  ]\

\[ (W_{\psi} f) (b,a) = \lbrace f, \psi _{b,a} \rbrace  \]
such that 
\begin{itemize}
\item time window width is $\Delta _\psi$
\item center at $t^\ast$
\item window at $ [ b + at^\ast -a\Delta_\psi,  b + at^\ast + a\Delta_\psi ] $
\item Window width $\alpha$
\end{itemize}

Reference page 61 Frequency Window

The IWT $W_\psi f$ also gives local information of $\hat {f}$ with a frequency window:
\[ [ \frac{\omega^\ast }{a} - \frac {1}{a} \Delta _\hat{\psi} , \frac{\omega^\ast }{a} + \frac {1}{a} \Delta _\hat{\psi}  ] \]
The frequency window as a  frequency band (or octave) with center-frequency $\frac{\omega^\ast }{a}$ and a bandwidth of $\frac {2\Delta \hat{\psi}}{a} $.  The ratio is independent of scaling which is a constant Q filtering.  

In other words, the IWT is defined as the inner product of a function and the wavelet, and this inner product is said to have the property of ``Constant Q Filtering''.   

Reference Theorem 3.10 page 62:


The IWT has a rewrite which comes from the following defines:
\[ \psi _{b,a} (t) = |a| ^{-1/2} \psi (\frac{t-b}{a}) \]
\[ ( \omega _\psi ) ( b, a) = \lbrace f, \psi_{b,a} \]

In other words IWT is defined as a the inner product of a function and the wavelet.

Such an inner product has the property of ``constant-q filtering''

The fact is the ratio of the center frequency to bandwidth is independent of ``a''. 

Reference Page 61 - 62
Theorem : ``Let \psi be a basic wavelet which defines an IWT $W_\psi$.  Then 

\[ C_\psi \lbrace f,g \rbrace \int ^\infty _{-\infty} \int ^\infty _{-\infty} [ W_\psi f) (b,a) \bar {(W_\psi g)(b,a)} ] \frac{da}{a^2} db\]
$\forall f,g \in L^2 (R)$ Furthermore, $\forall f \in L^2 (R)$ and $x\in R$ at which $f$ is continuous,

\[ f(x) \frac {1}{C_\psi } \int ^\infty _{-\infty} \int ^\infty _{-\infty} [(W_\psi f)(b,a)] \psi_{b,a} (x) \frac{da}{a} db \]
$\psi _{b,a} = |a| ^{-1/2} \psi (\frac {t-b}{a}) $  ''

Question is what does this theorem mean? 

Typical Signal Analysis
\[\omega >0 \]

\[ \omega = \frac{\omega ^\ast }{a} \]

such that 
\begin{itemize}
\item $\omega ^\ast$ is the center of $\hat{\psi}$
\item $\psi ^\ast > 0 $
\item $a > 0$
\end{itemize}

This is applied to the IWT $(W_\psi f)(b,a) $ 

The extra condition is 
\[  \int ^\infty _0 \frac{|\hat{\psi}(\omega)|}{\omega} d\omega = \int ^\infty _0 \frac{|\hat{-\psi}(\omega)|^2}{\omega} d\omega = \frac{1}{2} C_\psi < \infty \]

Theorem: Let $\psi$ be a basic wavelet that satisfies the above condition, then 
\[ \frac{1}{2}C_\psi \lbrace f,g \rbrace \int ^\infty _{-\infty} [ \int ^\infty _{-\infty} W_\psi f) (b,a) \bar {(W_\psi g)(b,a)} db] \frac{da}{a^2} \]

$\forall f,g \in L^2 (R)$.  Furthermore, $\forall f \in L^2(R)$ and $x \in R$ at which $f$ is continuous:
\[ f(x) = \frac{2}{C_\psi} \int ^\infty _ 0 [ \int ^infty _0 (W_\psi f) (b, a) \psi_{b,a} (x) db ] \frac{da}{a^2} \]
such that $\psi _{b,a} (t) = |a|^{-1/2} \psi (\frac {t-b}{a} )$.

\subsection {Dyadic Wavelet and Inversion }
Signal analysis \_ partition of disjoint frequency bands binary partitions.

\[\ast (0,\infty) = \cup ^\infty _{j=-\infty} ( 2^j \Delta_{\hat{\psi}} , 2^{j+1} \Delta_{\hat{\psi}} ]  \]

such that 
\begin{itemize}
\item $\Delta _{\hat{psi}} > 0$ is the radius of the Fourier Transform of a basic wavelet
\item $\hat{\psi}$ is the Fourier Transform of the basic wavelet 
\item $\psi$ is the basic wavelet
\end{itemize}


A phase shift of $\psi$ by $\alpha$ is equivalent to the forward shift in frequency of $\hat{\psi}$ by the $\alpha$.  
\[ \psi ^0 (t) - e^{i\alpha t} \psi (t) \to \hat{\psi} ^0 (\omega) = \hat{\psi} (\omega - \alpha)$

Note page 79 Theorem 3.27

  


 \end{document}
