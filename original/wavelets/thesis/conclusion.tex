%These following items shall be accomplished by this thesis:  testing matrix multiplication and testing partial differential equations with the wavelet transform as a preconditioning function.  The hypothesis for matrix multiplication is that a sparser matrix should emerge and therefore should be easier to multiply.  In case of the partial differential equation, the hypothesis is that  a better conditioned matrix should emerge.
%\begin{enumerate}
%\item Matrix Multiplication
%\item Wavelet Based Partial Differential Equation
%\end{enumerate}

%In order to satisfy the matrix multiplication component, a prototype shall be developed.  This prototype shall multiply two matrices together.  Several wavelet bases shall be used to condition the matrix.  The results shall be tested for their correctness.  

%For the wavelet based partial differential equation, another prototype shall be developed.  This prototype shall solve the PDE within the standard implicit method to provide means of comparison.  A few wavelet basis function shall be used to test their effectiveness in solving PDE.  Also, a few multi-resolution methods shall be used to test their effectiveness.  
The production of this thesis yielded three categories of results.  One category was prototypes for the three forms of multi-resolution for wavelet transforms.  Second, a proof was generate that satisfactorily shows that matrix multiplication is sound for the Haar Wavelet Transform with $\psi^n$ expansion.  Third,   empirical results were obtained showing the fidelity of these multiplications.  

For each of these categories, three segments were the sub sections for each one.  Those sub-sections were the three different forms of multi-resolution for the wavelet transform.  For these categories, a strong foundation was shown in the overview, clearly showing what the wavelet transform is and how it can be used.  

In conclusion, it is clear that there is a trade off between full wavelet transform expansion and sparse threshold for maintaining an acceptably precise result from matrix multiplication.  This thesis may further be used to explore other wavelet basis functions in matrix multiplication.  Also, it may be used to explore other numerical methods.   Furthermore, it may be interesting to explore wavelets a preconditioning agent for comparing Strassen versus sparse matrix multiplication techniques.  However, it is sufficient for this publication to present these methods of wavelet transform, the proof of the Haar Wavelet Transform as a preconditioning agent for matrix multiplication, and empirical results this agent when used in sparse matrix multiplication.  