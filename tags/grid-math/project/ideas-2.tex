\documentclass[11pt]{article}
\usepackage{graphicx}
\usepackage{amssymb}
\usepackage{epstopdf}
\DeclareGraphicsRule{.tif}{png}{.png}{`convert #1 `dirname #1`/`basename #1 .tif`.png}

\textwidth = 6.5 in
\textheight = 9 in
\oddsidemargin = 0.0 in
\evensidemargin = 0.0 in
\topmargin = 0.0 in
\headheight = 0.0 in
\headsep = 0.0 in
\parskip = 0.2in
\parindent = 0.0in

\newtheorem{theorem}{Theorem}
\newtheorem{corollary}[theorem]{Corollary}
\newtheorem{definition}{Definition}

\title{Project for Applied Numerical Analysis II: The Choices there of}
\author{Dan Beatty}
\begin{document}
\maketitle

In a nutshell, there are many components what I am wanting my dissertation to cover.  Scientific computing with numerical methods, knowledge base representation, service oriented computing and distributed computing are three areas that encompass the research.  Examples with the Sloan Digital Sky Survery, Geographical Information Systems, Weather modeling, and seismology are examples of the applicable benefactors for my work.  But that statement requires a definition of the work.  

I would like to focus scientific principles of distributed computing in service/ grid environment where I can treat all member computers as one large computer.   Furthermore, I would like for this system to have powerful numerical analysis capabilities, use those capabilities in probing vast knowledge, and test that knowledge in robust way such that the resources at hand intuitively work collectively with these tools.  

In this project, I am focusing some specific examples.   In particular I am linking xGrid to globus to provide the distribution and scheduling components of these services.  In addition, many of these scientific libraries are being ported to Objective-C to be used as network accessible services.   Furthermore, specific computing domains are to be linked to take advantage of areas where bandwidth is better.   



 \end{document} 