\documentclass[11pt]{article}
\usepackage{graphicx}
\usepackage{amssymb}
\usepackage{epstopdf}
\usepackage{doublespace}
\DeclareGraphicsRule{.tif}{png}{.png}{`convert #1 `basename #1 .tif`.png}

\textwidth = 6.5 in
\textheight = 9 in
\oddsidemargin = 0.0 in
\evensidemargin = 0.0 in
\topmargin = 0.0 in
\headheight = 0.0 in
\headsep = 0.0 in
\parskip = 0.2in
\parindent = 0.0in

\newtheorem{theorem}{Theorem}
\newtheorem{corollary}[theorem]{Corollary}
\newtheorem{definition}{Definition}

\title{Critic on Tools to detect non-Gaussianity in non-standard cosmological models}
\author{Daniel D. Beatty}
\begin{document}
\maketitle

Reason for wavelet concentration:
\begin{enumerate}
\item Space-frequency localization
\item Demonstration in CMB and COBE-DMR data analysis
\item Note spherical wavelets (Tenorio)
\end{enumerate}

Experiments with spherical wavelets on COBE-DMR data in HEALPix pixelization.  They were looking for Gaussianity of data by using 
\begin{itemize}
\item Spherical Haar Wavelet (SHW)
\item Spherical Mexican Hat Wavelet (SMHW)
\end{itemize}

They measured performance of these wavelets for discriminating between standard inflationary models and non-Gaussian models.  

For astronomical data, these non-Gaussian effects are said to be attributes of the following:
\begin{itemize}
\item Cosmic defects
\item Linear defects
\item hot spots 
\item cold spots
\end{itemize}

Other examples referenced 
\begin{itemize}
\item Salopek and Spergel
\item Gangui et al
\end{itemize}

$\phi(\vec{n}) = \phi (\vec{n}) +f_{i,l}(\phi^2_L (n) - <\phi^2,n>) $

Statement: The model has been the gravitational potential $ \phi(\vec{n})$ includes now a quadratic term with amplitude regulated by the so called non-linear coupling parameters such that:
\begin{itemize}
\item $\phi_L$ is the linear part of the gravitational potential, $f_{n,l}$
\item $f_{n,l}$ is the nonlinear coupling parameter controlling the amount of non-Gaussianity introduced.  
\item $<>$ is a volume average.
\end{itemize}

Cayon et. al. applied a wavelet space statistic model to simulations of slow-roll inflation model to quantify the probability of non-Gaussiantity detections used by the above model:

Simple Questions:
\begin{enumerate}
\item What are the SMHW and SHW?
\item What are the properties of SMHW and SHW?
\item What is the performance of these methods in Gaussian conditions?  
\item What is the performance of these methods in non-Gaussian conditions?
\end{enumerate}

From Section 2:  The Spherical Wavelets:

%Page 63

Claim: The Mexican Hat Wavelet family has been successfully used to extract point sources from CMB maps.

\section {The Mexican Hat on $S^2$ }
Group Theory approach by Antoine and Vanderghegnst have extended to the wavelet on $S^2$ based on these following properties:
\begin{itemize}
\item The basic function is a compensated filter
\item Translations
\item Dialations
\item Euclidean Limits for small angles
\end{itemize}

Conclusions from expansion:
The stereo-graphic projection on the sphere is the appropriate one to translate the mentioned properties from plane to the sphere.  

Projection definition $(\vec{x} \to (\theta, \phi))$

$x_1 = 2 \tan \frac{\theta}{2} \cos \phi$

$x_2 = 2 \tan \frac{\theta}{2} \sin \phi $

where 
$ (\theta , \phi)$ represent polar coordinates in $S^2$, $(y=\equiv 2 \tan \frac{\phi}{2}, \phi)$ are polar coordinates in the tangent plane to the North Pole.  

What does this do for an isotropic wavelet $\psi(x;R)$?  

$\psi_s (\phi, R)\alpha (\cos \frac{\theta}{2} )^{-4} \psi(x\equiv 2 \tan \frac{\theta}{2}; R) $

Analysis: A function on a sphere $f(\theta , \phi)$

Definition: Linear operation continuous wavelet transform with respect to $\psi_s (\theta, R)$ is defined 

$\tilde{w} (\vec{X}, R) = \int d\theta\prime \sin \theta\prime \hat{f}(\vec{x} + \vec{u}) \psi_s (\theta, R)$

$\vec{x} \equiv 2 \tan \frac{\theta}{2}(\cos\theta, \sin\theta) $

$\vec{u} \equiv 2 \tan \frac{\theta\prime}{2}(\cos\theta\prime, \sin\theta\prime) $

$\hat{f}(\vec{x}) = f(\theta, \phi)$

$w = (\theta , \psi; r) \equiv \tilde{w}(\vec{x},R)$ are wavelet coefficients dependent on 3 parameters.

Synthesis 
The equation $\psi = (\theta\prime;R)$ leads to the following reconstruction formula.

$f(\theta , \phi) = \bar{f} = \frac{1}{C_\psi} \int d\theta\prime d\phi\prime \sin\theta\prime \frac{dR}{R} (\vec{x} +\vec{\mu} , \vec{R} ) $

Example of Mexican Hat

$\psi (\theta, R) = \frac{1}{\sqrt{2\pi} RN} [1 + (\frac{1}{2})^2]^2 [2-(\frac{y}{2})^2]e^{-\frac{y^2}{2R^2]}}$

$N(R) \equiv +\sqrt{1 + \frac{R^2}{2} + \frac{R^4}{4}} $

$y \equiv 2\tan\frac{\theta}{2}$


\section {Spherical Haar Wavelet}
Claims 
\begin{enumerate}
\item Introduced by Sweldens with general planar Haar Wavelet to a pixelized sphere.
\item Orthogonal 
\item Adapted to pixelization of the sky 
\item ? Must be hierarchical ?
\item ?Dialation and translation capability lost ?
\item SHW decomposition is based on one scaling $\phi_{j,k}$ and three wavelet functions $\psi_{m,j,k}$ at each resolution j and position on the grid k. 
\item$ N_{side} =2^{s-1}$
\item The total number of pixels for a level j, w/ in an area $\mu_j$ is given by $n_j = 12 \times 4^{J-1}$
\item Each pixel resolution j $(S_{j,k})$ is divided into four pixels $(S_{j+1,k_0}, S_{j+1,k_1}, S_{j+1,k_2}, S_{j+1,k_3})$
\item Their tool: HEALPix
\end{enumerate}

Key equations:
\begin{itemize}
\item $\phi_{j,k} (x) = 
\left(
\begin{array}{ccc}
  & 1  & if\ x\in S_{j,k}  \\
  & 0  & otherwise  
\end{array}
\right)
$
\item $\psi_{1,j,k} = \frac {\phi_{j+1,k_0} + \phi_{j+1,k_2} - \phi_{j+1, k_1} - \phi_{j+1,k_3}}{4\mu_{j+1}}$
\item $\psi_{2,j,k} = \frac {\phi_{j+1,k_0} + \phi_{j+1,k_1} - \phi_{j+1, k_2} - \phi_{j+1,k_3}}{4\mu_{j+1}}$
\item $\psi_{1,j,k} = \frac {\phi_{j+1,k_0} + \phi_{j+1,k_3} - \phi_{j+1, k_2} - \phi_{j+1,k_1}}{4\mu_{j+1}}$
\end{itemize}

such that 
$k_0$, $k_1$, $k_2$, $k_3$ are the four pixels at resolution level j+1 into which k at level j is divided.  

Note similarity to 2-D average, vertical, horizontal and diagonal.  

Wavelet coefficients at level j can be obtained from the four corresponding approximation at level j+1, $\lambda _{j+1,k_i}$  .  

\begin{itemize}
\item $\lambda_{j,k} = \frac{1}{4} \sum\limits^3_{i=0} \lambda_{j+1,k_i} $

\item $\gamma_{1,j,k} = \mu_{j+1} (\lambda_{j+1,k_0} + \lambda_{j+1,k_2} -\lambda_{j+1,k_1} - \lambda{j+1,k_3})$

\item $\gamma_{2,j,k} = \mu_{j+1} (\lambda_{j+1,k_0} + \lambda_{j+1,k_1} -\lambda_{j+1,k_2} - \lambda{j+1,k_3})$


\item $\gamma_{3,j,k} = \mu_{j+1} (\lambda_{j+1,k_0} + \lambda_{j+1,k_3} -\lambda_{j+1,k_1} - \lambda{j+1,k_2})$
\end {itemize}

Question: Where do $ \lambda_{j+1,k_0}$ , $\lambda_{j+1,k_1}$  , $\lambda_{j+1, k_2}$ , and $\lambda_{j+1, k_2}$ come from.

According to the paper, the generation of coefficients starts with the original map j=J, for which the coefficients are defined by original values a pixel k.  

Questions:  
\begin{itemize}
\item What is this division which is refered to?  
\item Is this similar to 1-D and 2-D in the fact that the elements ahead or behind the original element are used to define the next resolution components?  It would make sense for a 3-D world based on spherical coordinates.  
\end{itemize}
%end page 66 (sections 1 and 2).





 \end{document}