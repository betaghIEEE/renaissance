\documentclass[11pt]{article}
\usepackage{graphicx}
\usepackage{amssymb}
\usepackage{epstopdf}
\DeclareGraphicsRule{.tif}{png}{.png}{`convert #1 `basename #1 .tif`.png}

\textwidth = 6.5 in
\textheight = 9 in
\oddsidemargin = 0.0 in
\evensidemargin = 0.0 in
\topmargin = 0.0 in
\headheight = 0.0 in
\headsep = 0.0 in
\parskip = 0.2in
\parindent = 0.0in

\newtheorem{theorem}{Theorem}
\newtheorem{corollary}[theorem]{Corollary}
\newtheorem{definition}{Definition}

\title{Spring 2003: Wavelets In Review}
\author{Daniel D. Beatty}
\begin{document}
\maketitle

Spring 2003 started by producing and testing a prototype wavelet engine based on study from Fall 2002.  This engine was document became "Simple Wavelets with Convolution" whose final for was produced April 14, 2003.  The source code was indexed and referenced from the Queen Muppet Machine (mspiggy.cs.ttu.edu) one week prior.  

During the week of the 14th, a review was conducted on Gregory Belykin's ``On wavelet-based algorithms for solving differential equations'' and ``Wavelets and Fast Numerical Algorithms.''  

Another author reviewed was Wim Sweldon and his ``Spherical Wavelets: Efficently Representing Functions on the Sphere.''  Sweldon's work became of interested after reviewing "Tools to detect non-Gaussianity in non-standard cosmological models" by J.E. Gallegos, E. Martinez-Gonzalez, F. Argueso,  L. Cayon, and J.L. Sanz. Both were reviewed May 2, 2003.  

May 1-2, 2003 presentation of wavelets and their possible use in astro-physics and the Sloan Digital Sky Survey was discussed with Dr. Stoughton.  One of the directions of immediate concern is compression of the SDSS data release, and reproduction of sections of sky to observers and researchers in real time.  Also of concern are useful tools astronomers and astro-physicist alike.  

Details of these research endeavors shall be expounded upon for good review.  It is not the case that all questions have been answered.  In fact, these reviews have produced more questions which have yet to have any answers at all.  Both known data and questions are presented.  


%April 3, 2003  

%Completed Vector-to-matrrix 2-D Wavelet Transform
%Added more information to thesis
%Added code to code browser http://mspiggy.cs.ttu.edu/wavelxr/source
%New Site: http://mspiggy.cs.ttu.edu/wavelet
%New personal site: http://mspiggy.cs.ttu.edu/~danbgood
%April 8, 2003
%Produced Multi-Resolution 2-D wavelet transform and inverse. 
%Applied Wavelet Threshold on triple resolved wavelet image
%Updated documentation
%April 16, 2003
%Proofed Section I of thesis proposal
%Read and made notes on sections 1-4 of AMS93 paper.
%April 24, 2003
%Placed on request reference material for inter-library loan.
%Reviewed G. Beylkin's "Wavelet's and Fast Numerical Algorithms"
%Reviewed first two sections of "Tools to detect non-Gaussianity            in non-standard cosmological models"
%May 2, 2003
%Reviewed 1st 3 sections in "Tools to detect non-Gaussiantity            in non-standard cosmological models"
%Reviewed 1st 3 sections in Wim Swelden's paper on spherical wavelets.
%In process of reviewing Beylkin's paper on Diff. Eq. in wavelet space.
%May 9, 2003
%Done:
%Reviewed Beylkins Boundary Value Problem paper up to section 6.
%Still reviewing Swelden's paper.
%May 13, 2003
%Done: Reviewed Fast Approximate Factor Analysis
%Still to be done:  
%Finish Swelden's paper
%Finish review Beylkin's BVP paper
%Review OpenGL for use as a tool and make a few sample examples
%Place into order the wavelets homepage. 

 \end{document}