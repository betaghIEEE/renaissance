
\documentclass{article}
%%%%%%%%%%%%%%%%%%%%%%%%%%%%%%%%%%%%%%%%%%%%%%%%%%%%%%%%%%%%%%%%%%%%%%%%%%%%%%%%%%%%%%%%%%%%%%%%%%%%%%%%%%%%%%%%%%%%%%%%%%%%
\usepackage{graphicx}
\usepackage{amsmath}

%TCIDATA{OutputFilter=LATEX.DLL}
%TCIDATA{Created=Thu Feb 13 17:14:08 2003}
%TCIDATA{LastRevised=Fri Feb 14 00:46:44 2003}
%TCIDATA{<META NAME="GraphicsSave" CONTENT="32">}
%TCIDATA{<META NAME="DocumentShell" CONTENT="General\Blank Document">}
%TCIDATA{CSTFile=LaTeX article (bright).cst}

\newtheorem{theorem}{Theorem}
\newtheorem{acknowledgement}[theorem]{Acknowledgement}
\newtheorem{algorithm}[theorem]{Algorithm}
\newtheorem{axiom}[theorem]{Axiom}
\newtheorem{case}[theorem]{Case}
\newtheorem{claim}[theorem]{Claim}
\newtheorem{conclusion}[theorem]{Conclusion}
\newtheorem{condition}[theorem]{Condition}
\newtheorem{conjecture}[theorem]{Conjecture}
\newtheorem{corollary}[theorem]{Corollary}
\newtheorem{criterion}[theorem]{Criterion}
\newtheorem{definition}[theorem]{Definition}
\newtheorem{example}[theorem]{Example}
\newtheorem{exercise}[theorem]{Exercise}
\newtheorem{lemma}[theorem]{Lemma}
\newtheorem{notation}[theorem]{Notation}
\newtheorem{problem}[theorem]{Problem}
\newtheorem{proposition}[theorem]{Proposition}
\newtheorem{remark}[theorem]{Remark}
\newtheorem{solution}[theorem]{Solution}
\newtheorem{summary}[theorem]{Summary}
\newenvironment{proof}[1][Proof]{\textbf{#1.} }{\ \rule{0.5em}{0.5em}}
\input{tcilatex}

\begin{document}


\section{Haar Wavelet Transform Class:}

\bigskip The Haar Wavelet Transform class has the purpose of taking in a
signal and outputing the signal processed by the Haar Wavelet Transform. \
It is dependent on the convolution function for computation. \ It is also
dependent on the haar function generator to establish a Haar Scaling and
Wavelet vector. \ For practical I/O, a result plotter is also necessary. \ 

\bigskip All of the above can be accommodated within one class. \ One other
class that is necessary is the myVector class. \ This class provides a
simple, yet practical data type for the computation. \ It can be allocated
and deallocated such that its size is appropriate for the task. \ 

It is projected that the two-dimensional version will simply add a matrix
manipulator. \ What such a manipulator extracts rows and columns into
myVector structures such that the computation to procede on each row and
column.. \ 

\subsection{\protect\bigskip One Dimensional Wavelet Transform}

The primary function in the Wavelet transform class is the one dimensional
wavelet transform (hWaveXform). \ More than one form of this function could
possible exist; however, for simplisty only one was produced. \ It takes two
myVector classes (input and output). \ It produces a haar difference and
scaling vector, as well as a $R_{A}$, $R_{D}$, $T$, $F$ and $S$ myVector
classes for use within the function. \ The algorithm is as follows

$S=input$

$l=$ is the size of S (or the input)

$\forall i\in \lbrack 0,l)$

\qquad $R_{A}=S\ast H_{A}$

\qquad $R_{D}=S\ast H_{D}$

\qquad T = join ($R_{A}$,$R_{D}$)

\qquad F = evenSplit (T)

\qquad ForceInsert (F, output)

\qquad end = floor ( $\frac{l}{2^{i+1}}$)

\qquad Extract (output, S, 0, end)

\bigskip

The Extract procedure is to take values from output up to the index end, and
make S a copy of that vector. \ Thus S is called the resolution vector. \ T
is always twice the size of the working S vector. \ F is always half. \ The
resolution vector decreases by half each iteration, and starts at the same
size as the input.

\bigskip

\subsection{Join Procedure}

The join procedure is rather simple. \ Produce a result whose size is the
size of the two sources combined, and whose elements are those of the two
input vectors. \ 

Input:

\qquad myVector left, right

Output:

\qquad myVector Result

Algorithm

\qquad result.deallocate

\qquad $s_{l\,}=size(left)$

\qquad $s_{r}=size(right)$

\qquad $s=s_{l}+s_{r}$

\qquad result.allocate(s)

\qquad $\forall i\in \lbrack 0,s_{l})$

\qquad \qquad result[i]=left[i]

\qquad $\forall i\in \lbrack s_{l},s)$

\qquad \qquad result[i]= right$[i-s_{l}]$

\subsection{Procedure: Even Split}

This procedure is simply a special type for condensing procedure. \ Simply
put, this procedure makes the result half the size of the original input. \
Both the input and result are myVector classes. \ The characteristic of the
elements of the result is:

\qquad result[i] = input[2*i]

\subsubsection{Input:}

\qquad myVector s

\subsubsection{Output:}

\qquad myVector r

\subsubsection{Algorithm}

\qquad l = ceil (s.size/2)

\qquad r.allocate(l)

\qquad $\forall i\in \lbrack 0,l)$

\qquad \qquad $r_{i}=s_{i\ast 2}$

\bigskip 

\bigskip 

\subsection{Procedure: Force Insert}

\bigskip Purpose: To insert one myVector ,s, into a larger myVector, r. \
The constraint is that r be larger than s. \ As long as this is the case,
then forced insertion can occur. \ Special cases can include start and end
points. \ In which case, the start and end points must be within the
specified range.

\subsubsection{Input: }

\qquad myVector s, r

\subsubsection{Output:}

\qquad myVector r

\subsubsection{Algorithm:}

\qquad $l_{1}=s.size$

\qquad $l_{2}=r.size$

\qquad if ($l_{1}<l_{2}$) and (start \TEXTsymbol{<} end \TEXTsymbol{<} $l_{2}
$)

\qquad \qquad $\forall i\in \lbrack start,end]$

\qquad \qquad \qquad $r_{i}=s_{i}$

\bigskip 

\subsection{Procedure: Extract}

Purpose: To produce a new or replace a myVector whose length is that of the
section to copied and extracted. \ The point behind this procedure is
provide allow a wavelet transform to be performed on a segment of data, and
keep the procedure the same each resolution. \ 

\bigskip 

\subsubsection{Input:}

The inputs for the wavelet Xform -\TEXTsymbol{>} extraction procedure are as
follows:

\qquad myVector S, 

\qquad integer a \ 

\qquad integer b \ 

These names are arbitray. \ The variables a and b specify the start and end
points respectively. \ Obviously, there is an inequality relationship here.

\qquad $0\leq a\leq b\leq S.size$

\bigskip 

\subsubsection{Output:}

The output of the wavelet Xform -\TEXTsymbol{>} extraction procedure is
simply:

\qquad myVector R

The elements of R are a copy of the elements of S from index a to index b.

\bigskip 

\subsubsection{Algorithm}

\qquad if ( 0 \TEXTsymbol{<} a \TEXTsymbol{<} b \TEXTsymbol{<} S.size)

\qquad \qquad R.deallocate

\qquad \qquad l = b - a +1

\qquad \qquad R.allocate(l)

\qquad \qquad $\forall i\in \lbrack a,b]$

\qquad \qquad \qquad $R_{i-a}=S_{i}$

\bigskip 

\qquad 

\end{document}
