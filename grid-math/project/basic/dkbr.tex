\documentclass[11pt]{article}
\usepackage{graphicx}
\usepackage{amssymb}
\usepackage{epstopdf}
\DeclareGraphicsRule{.tif}{png}{.png}{`convert #1 `dirname #1`/`basename #1 .tif`.png}

\textwidth = 6.5 in
\textheight = 9 in
\oddsidemargin = 0.0 in
\evensidemargin = 0.0 in
\topmargin = 0.0 in
\headheight = 0.0 in
\headsep = 0.0 in
\parskip = 0.2in
\parindent = 0.0in

\newtheorem{theorem}{Theorem}
\newtheorem{corollary}[theorem]{Corollary}
\newtheorem{definition}{Definition}

\title{Concept of a Grid Scientific Knowledge Base}
\author{Daniel Beatty}
\begin{document}
\maketitle

This project is a feasibility study on the concept of a Scientific  Knowledge Representation and Computational Service.  
%This field of this research is a union of distributed computing, scientific computing, and computational intelligence represented by knowledge base representation.   
The concept is produce a Grid/Distributed Knowledge Base  which is a service which considers the numerical libraries as services as well.    One characteristic of this Scientific Knowledge Base Service is the ability to use multi-level resource and data management to facilitate its goal.  If these distributed systems can be treated as collective of machines or simply a large computer, then principles of a large computer comes into play.    The semantics of such an KB must implicitly translate or contribute to some service oriented paradigm in a way to optimize use of these collectives.  

The first steps in generating such a Scientific Knowledge Representation and Computing Service is to define and demonstrate what a multi-level resource is.  Next is to demonstrate an application using this service.  Finally, show how such an application can be improved both algorithmically, and conceptually in this form.  

Chosen for these exercises are xGrid for the cluster manager,  Globus for the cluster interconnect, and SDSS for the application.   xGrid requires a simple API that can allow either cluster interconnection or application use of its resource fluently.  The SDSS project also requires its astro-tools converted to a framework which can be mobile and published as a service.  A framework is a collection of objects organized in a hierarchy.  

\section{What are XGrid or Globus?}
XGrid and Globus are middle ware packages which some of the basic functions of a Network Operating System.  Neither one of these packages completely implements enough functionality to be considered a Network Operating System on basis of simple operating systems.  However, there is process management, storage management, and communication methods for coordinating nodes on a network in some basic way.  

What does this have to do with a Scientific Knowledge Base and Computation Service?  The middle-ware can still be thought of in terms of a Turing machine (even if incomplete).   Furthermore, semantics of such a Knowledge Representation and Reasoning system or its more basic form a Knowledge Base can be defined in terms that be executed in this middle ware.    

Also, distributed jobs and published services are a corner stone of Globus.  Distributing tasks, published resources and discovering resources are a corner stone of XGrid.    The scale of the two allow for powerful conceptions of a KBR to exist.   

 \end{document} 